\documentclass[11pt,letterpaper]{article}

% Packages
\usepackage[margin=1in]{geometry}
\usepackage{graphicx}
\usepackage{hyperref}
\usepackage[normalem]{ulem}
\usepackage{soul}
\usepackage{parskip}
\usepackage{titlesec}
\usepackage{datetime}
\usepackage{enumitem}

% Hyperref setup
\hypersetup{
    colorlinks=true,
    linkcolor=blue,
    urlcolor=blue,
    citecolor=blue
}

% Format sections to be inline with text and numbered
\titleformat{\section}[runin]
  {\normalfont\bfseries}
  {\thesection.}
  {0.5em}
  {}
\titlespacing{\section}
  {0pt}
  {1em}
  {0.5em}

% Remove subsection from TOC and make it similar to section
\titleformat{\subsection}[runin]
  {\normalfont\bfseries}
  {\thesubsection.}
  {0.5em}
  {}
\titlespacing{\subsection}
  {0pt}
  {1em}
  {0.5em}

% Date format for last revised
\newdateformat{revisedate}{\THEDAY\ \monthname[\THEMONTH] \THEYEAR}

\begin{document}

% Header with logos
\noindent
\includegraphics[width=2.11111in,height=0.43056in]{cu-logo.png}\hfill
\includegraphics[width=1.42292in,height=0.49931in]{adcl-logo.png}

\vspace{1em}

\begin{center}
\textbf{\Large ADCL PhD Advising Agreement}

\vspace{0.5em}

Last Revised: \revisedate\today
\end{center}

\vspace{1em}

Welcome to the ADCL! I am excited to begin the PhD journey with you and grateful that you are joining our lab. PhD students are at the core of all of our research contributions, so your role is very important. This document provides perspective on expectations related to advising. It is not a formal contract, but a set of guidelines about what we can expect from each other.

\section{What is a PhD?} A PhD has two objectives:
\begin{enumerate}[nosep]
    \item to add to human knowledge by contributing to the ADCL's research, and
    \item to gain the knowledge and skills you will need for your future career.
\end{enumerate}
Each student's PhD experience is different. While all of the guidelines in this document are flexible, I approach every interaction with the expectation that you are committed to the two objectives above. 
Concretely, you are expected to fulfill the requirements in the department graduate \href{https://www.colorado.edu/aerospace/current-students/graduates/curriculum/graduate-student-handbooks}{\ul{handbook}} in a timely manner.

By the end of your PhD, you will have discovered things that no one else in the world knew previously, but there will be many challenges on the way to that point. You have the opportunity to devote this time of your life to learning and discovery, and with me and the other ADCL members on your team, the journey itself can be exciting and rewarding.

\section{Your Role:} A PhD is not only as the last step of your education, but also an entry level job in research. You are a member of the CU academic community, and PhD students are the primary individual contributors to the research mission. This is an important role in society and it is a privilege with responsibilities to occupy it. It is also a formative process. I expect you to take responsibility for achieving the goals above, do your work with excellence, and be resilient when setbacks happen. When questions or challenges arise, I expect you to respond creatively and take initiative to solve them.

For most students, a PhD is a considerably more open-ended and less well-defined task than their previous education and work experiences. There are few fixed rules, and your role can best be described as that of a \textbf{knowledge entrepreneur}. Your task is to discover and marshal resources, including yourself, your colleagues, the academic literature, my advice, and others, to create and share new knowledge. I want to see you succeed at this, and I am confident that you can.

\section{Publications:} The most important tangible ``product'' of our research is publications in peer-reviewed journals and conferences. This makes our findings widely available and provides a ``stamp of approval'' from the research community, through the review process, that our research is a valid, substantial contribution. During your PhD, you should plan to submit \textbf{at least 3 papers} to journals or high quality conferences (those that have a rigorous peer review process, less than about 30\% acceptance rate, and recognition in the field as a final publication venue), though the specific contributions of each PhD student will vary. Learning to write these papers is a challenge that requires practice, so I recommend writing and submitting early and often.

\section{Communication:} During the PhD, you and I will often have complementary sets of knowledge. For example, you will usually understand the low-level details of the problem you are working on and your time constraints, while I have a better high-level picture of the research sponsor's goals and the research landscape. Because of this, communication is crucial.

The best way to communicate about complex ideas is through in-person meetings (see the meetings section). If you have a short question, you can always knock on my door in AERO 263, and I will answer if I have time. If the communication is about something straightforward, such as an administrative task, you can use electronic communication. If you send me an email, I try to guarantee that I will respond eventually if a response is needed. The lab also has a slack channel for brief communication. You can think about the difference between email and slack as the difference between the TCP and UDP protocols. You can be sure that I will (eventually) read and process an email; you may receive a quicker response to something simple over slack, but there is not a guarantee that I will process it. If there is an urgent need or we are traveling together at a conference, you can call or text me at 720-933-7799. I will not usually call or text you unless we are traveling together or in extraordinary circumstances, for example if you have missed a major deadline and I cannot contact you through email or slack.

\section{Funding and Work Expectations:} Nearly all PhD students are funded by a teaching assistantship (TA), graduate research assistantship (GRA), or fellowship. If you are eligible, during your first year, I will expect you to apply for the NSF GRFP, NDSEG, and possibly the Draper fellowships. These will give you extra flexibility to focus on research questions that you find interesting and allow us to accomplish more research goals as a lab. If you are on a GRA, at least 20 hours per week of your research time should be directed towards the research goals of the grant you are funded on and there may be specific deliverables that you need to direct your work towards. I will try to match you with research projects that you enjoy, but, as a GRA you may have to work on sponsored research that differs from your interests. We will work together to set a research agenda that progresses towards both your goals and the sponsor's as much as possible.

For a TA, you are likely to work with another faculty member. The expectation is that you work 20 hours per week with some week-to-week variation. If you feel the position is far exceeding those expectations, please talk to the faculty member(s) you are TA-ing with and/or me. While you are a TA, I expect you to continue making progress on your research, but understand productivity will be reduced. We will often use this mechanism as a way for you to have bridge funding within your PhD, which is why it's important that you prioritize research to avoid excessive timeline delays.

\subsection{Work Hours, Vacation / Time Off:} Outside of your TA/GRA appointment, there is not a formal requirement for  hours worked or vacation days. However, I expect that you are committed to pursuing the PhD objectives with excellence, and you must make adequate progress towards completing the PhD.

I support students going on vacation, as I believe it is important to rest. If you are taking more than a day or two off, I request you give me at least 3 weeks notice, and there may be certain times of the year that are difficult for travel, e.g. before critical deadlines or TA responsibilities. For emergencies, no notice is needed, but please let me know you are out so I don't worry.

Though there are no formal requirements for hours worked or vacation days, many students ask for guidance about how much they should work. In this case, I advise working hours equivalent to a demanding full-time job and taking the following amount of vacation:
\begin{itemize}[nosep]
\item University Holidays (individual days when classes are cancelled)
\item Winter Break: About 1 week of no work, 1 additional week of less-intense work
\item About 2 additional weeks (Often taken during Thanksgiving, Spring, and Summer Breaks)
\end{itemize}

\subsection{In-Person Work:} As a PhD student, you are expected to maintain a physical presence in the lab at CU. The most effective working environment can vary from person to person, but it is beneficial to cross paths with other students, so there is an expectation that you spend some time working in the AERO building. You are expected to attend all lab meetings and individual meetings in person, and, at a minimum, spend at least \textbf{two full days} (or four half days) per week physically in the lab. It is beneficial to spend more than this and many students work from their desk every day.

I am generally working from about 9am to about 5pm M-F, but I also work outside of normal business hours to support our lab's mission. I am generally on campus every work day and am happy to have you pop in if you need something.

% In keeping with the expectation of progress towards the two objectives of the PhD, I recommend students take approximately 2-3 weeks of vacation a year including spring break, plus university holidays when classes do not occur. This is in addition to break over the winter holiday. Over winter holiday, we tend to take about a week completely off from work, and work less intensely for another week. You can adjust this as best serves you as long as it falls within the University winter break. Note that if you are a TA, you need to be available the week before classes start and until grading is finalized (usually a few days after the final exam period). Please communicate with the faculty member(s) leading the class you are TA-ing for to solidify their expectations. With some discussion/approval, you may work remotely, if you are able to complete your research activities, not counting as ``vacation''. During vacation, normally that means not working at all (i.e., not answering emails -- though I may send emails because things are on my mind, I am not expecting you to check or respond). In certain instances, there may be a quick email for a deadline, but we will try to plan around deadlines before you go on vacation to avoid this.

\subsection{If You Are Sick:} Please do not come to the lab. Rest and get well. If you are not working due to illness for more than a day or two, please notify me.

\section{Regular Meetings:} We will have regular one-on-one meetings, usually once per week. For some larger projects, we will have a project meeting (e.g., with undergraduate research assistants, other advisors/collaborators, or other graduate students working on the same/similar project). We will also have regular group meetings.

In one-on-one meetings, you should bring a list of items to update on/discuss as an informal agenda. As you progress in your project, I expect you to be dictating the direction of these meetings. Developing a slide deck or collaborative document for weekly meetings can help anchor the discussion and provide a quick way of getting feedback on results. Since there are many projects going on in the lab, I may not remember some details of the project you are working on. Thus, it is helpful to bring figures and/or data that is formatted to be easy to read (printf and data frames are useful!) to our meetings. For the most part, the advisee should take notes during the meeting. I will also take notes of things that I need to do. If needed, don't hesitate to pause/interrupt me to take notes. At the end of the meeting, we should debrief on next steps and the plan for the upcoming week.

\section{Conference Travel and Presentations:} A key component to your research career is to attend conferences to learn about the field and present at conferences to disseminate our work. It is also important for career advancement.

Typically, conference travel expenses will be covered by a research grant, a fellowship, or some other means. It is important to recognize that frugal (but not austere!) spending on travel helps our funding go further, potentially creating more opportunities for additional conferences during the year. Further, some grants may restrict international travel due to budget constraints. I encourage my students to apply for fellowship opportunities through the University, College, and even the conference itself (e.g., by volunteering to help with the conference). ADCL students should look for opportunities to share resources to keep costs reasonable (e.g., sharing rental cars and lodging).

At conferences, I expect you to act professionally on behalf of our lab, be prepared, attend sessions (ask questions!), dress professionally, and try to make the most of the experience. If I also attend, I will do my best to help you navigate the conference and can help introduce you to other researchers, etc. Please do explore/enjoy the location of the conference, but remember that your priority is attending the conference. Many students add personal travel before or after the event. This is a great opportunity to see a new place while you are there. To do this, you will need to provide proof that flights are not more expensive on the days that you travel (check exact department guidelines), and you should not seek reimbursement for lodging or food on non-conference days. 

\section{Feedback from Me:} I will provide informal feedback in our one-on-one and group meetings. I will strive to provide constructive criticism; know that my feedback is always coming from a desire for you to succeed.

\section{Your Career Goals:} I sincerely want to see you succeed in whatever career path you would like to pursue. Please keep me updated on what your career plans are and how they may be changing. For students who have aspirations in academia, I can provide opportunities to understand what that job is like through grant writing, TAs, etc. Some students do an internship during their PhD, which can help build connections and visibility of what that career looks like. I want to help you find your dream job, so please feel free to bring this up as opportunities arise.

% \section{If You Make a Mistake} Please come to me and let's discuss. Mistakes happen during research. THIS IS NORMAL. This is how we learn.

\section{Graduate School Is Not Easy!} It's a marathon, and there will be ups and downs throughout the process. You may make a mistake or feel like an imposter or even a failure. THIS IS NORMAL. Please know you were brought into our research group for a good reason, and I am here to support you -- work hard and don't give up. You can do this!

\section{Boundaries:} The advisor-advisee relationship is a unique one. We will spend a lot of time together, travel together, and discover new things in research together. In a relationship like this (well, in any relationship really), it's healthy to have boundaries. I am happy to hear about about your life outside of the PhD, and I will sometimes share about my life outside of academia with you, but you are not expected to share anything you would prefer not to share, and I will not pry. If there is a personal struggle or circumstance preventing you from making progress, it can be helpful to communicate this so that we can make adjustments to your work to accommodate it, but there is no expectation to share this.

\section{Conflict Resolution:} If conflicts should arise, I suggest we attempt to resolve the issue informally. If you don't feel like I am hearing your concern or you feel uncomfortable raising it with me, you should first go your graduate advisor (if you don't need confidentiality). Other faculty are also helpful at giving advice. The \href{https://www.colorado.edu/ombuds/}{\ul{Ombuds}} office can act as a confidential resource for conflict resolution, along with assisting with mediation in some cases. If a resolution cannot be reached informally, you may consider additional avenues for your complaint. The Graduate School grievance process and procedures document includes information about jurisdiction for a variety of issues, and explains the process for grievances which fall under the purview of the Graduate School. \href{https://www.colorado.edu/graduateschool/current-students/campus-resources}{\ul{Resources}} related to conflict resolution and information on the \href{https://www.colorado.edu/graduateschool/current-students/graduate-school-policies-and-procedures}{\ul{grievance process}} can be found on the Graduate School website. Conflicts related to discrimination and harassment or sexual misconduct should be reported to the \href{https://www.colorado.edu/oiec/reporting-resolution-options}{\ul{Office of Institutional Equity and Compliance}}.

\section{Not Meeting Expectations:} If I assess you are not meeting expectations in a particular aspect of your PhD, I will try to help identify things that are preventing this through communication in our weekly meetings. If I assess that you are not meeting expectations on a long term basis, I will, at a minimum, communicate this in our yearly check-in. I want you to succeed personally, gain satisfaction from being a contributing member of our research group, and make progress towards whatever comes next in your career.

\section{If Your Research Interests Have Changed or You Would Not Like to Continue in Our Lab:} It is okay to let me know if your research interests have changed, or if you feel the working culture in this lab is not the best fit for you. Please feel free to come and discuss with me on this. We can sort out issues together and work on your next plan of action, next steps to succeed in the PhD program or your next career step. It is helpful to me if you can communicate this early so that I can smoothly transition personnel to continue meeting the goals of the ADCL's research mission.

\textbf{Please feel free to add any other points of concern below prior to initialing.}

\vspace{3em}

\noindent\rule{6cm}{0.4pt}\hfill\rule{6cm}{0.4pt}

\vspace{0.5em}

\noindent Student Initials\hfill Advisor Initials

\end{document}
